\documentclass[nomag*,a4paper,leqno,uplatex]{jsarticle}
\bibliographystyle{jplain}
\usepackage{nishida}

\numberwithin{thm}{subsection}
\numberwithin{equation}{subsection}

\usepackage[dvipdfm]{geometry}
\geometry{
  truedimen,
  mag    = 1000,
  top    = 25truemm,
  bottom = 25truemm,
  left   = 25truemm,
  right  = 25truemm
}

\title{\vspace{-2.5em}Sheaves on subanalytic sites セミナーノート\vspace{-1em}}
\date{\vspace{-1em}\today\vspace{-1.5em}}

\begin{document}
\maketitle
\tableofcontents
\addtocounter{section}{-1}
\section{Preface}
このノートでは,
L. Prelli,
Sheaves on Subanalytic Site \cite{book:Prelli}
を参考にして,
Subanalytic sitesや,
その上の層についてまとめる.
また, 必要に応じて,
Kashiwara-Schapira\cite{book:KS_sh}, \cite{book:KS_cat}
を参照する.

\section{Sheaves on sites}
\subfile{contents/1-1} % ind-sheavesからのsite入門
\subfile{contents/1-2} % subanalytic siteの入門
\subfile{contents/1-3} % R-constructible sheavesとの関係

\section{Appendix}
\subfile{contents/a-1} % subanalytic set
\subfile{contents/a-2} % constructible sheaves

\bibliography{ref}
\end{document}
