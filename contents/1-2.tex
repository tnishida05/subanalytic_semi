\documentclass[../main]{subfiles}

\begin{document}
\setcounter{section}{1}
\setcounter{subsection}{2} \addtocounter{subsection}{-1}
\subsection{Subanalytic sites and sheaves}
\begin{dfn}[subanalytic site]
  $\Op(X_{sa})$を$X$のsubanalyticな部分集合の圏とする.
  この圏には, 次のような位相が入る:

  $S\subset\Op(X_{sa})$が$U\in\Op(X_{sa})$の被覆であるとは,
  $X$の任意のコンパクト集合$K$に対して,
  ある有限部分集合$S_0\subset S$で,
  $K\cap (\cup_{V\in S_0} V) = K\cap U$
  となるものが存在する.

  このような$X_{sa}$をsubanalytic siteと言う.
  また, $U_{X_{sa}}$を,
  $\Op(X_{sa})\cap U$に$X_{sa}$の位相から誘導される位相を入れたものとする.
  一般に, $U_{sa}$と$U_{X_{sa}}$は異なる.
\end{dfn}

\begin{eg}
  $X = \mathbb{R}^2$, $U = \mathbb{R}^2\setminus\{0\}$とする.
  このとき, $V_n = \left\{ x\in\mathbb{R}^2 \relmiddle{|} |x|>\frac{1}{n} \right\}$とすると,
  $\{V_n\}_{n\in\mathbb{N}}\in\Cov(U_{sa})$であるが,
  $\{V_n\}_{n\in\mathbb{N}}\not\in\Cov(U_{X_{sa}})$である.
\end{eg}

\begin{dfn}
  $\Mod(k_{X_{sa}})$を$X_{sa}$上の層の圏とする.
\end{dfn}

\begin{dfn}
  $\Op^c(X_{sa})$で,
  $X$の相対コンパクトなsubanalytic開集合のなす圏とし,
  $X_{sa}$から誘導される位相を入れたものを$X_{sa}^c$と書く.
  $X_{sa}^c$上の前層(resp. 層)の圏を
  $\Psh(k_{X_{sa}^c})$(resp. $\Mod(k_{X_{sa}^c})$)と書く.
\end{dfn}

\begin{prop}
  $\Mod(k_{X_{sa}})$は, Grothendieck圏である
  i.e.
  生成元を持ち,
  small inductive limits と
  small filtrant inductive limits が完全となる圏である.
  特に, Grothendieck圏として,
  $\Mod(k_{X_{sa}})$は, enough injectiveな対象を持つ.
\end{prop}

\begin{prop}
  忘却関手$\Mod(k_{X_{sa}})\to\Mod(k_{X_{sa}^c})$は
  圏同値を与える.
\end{prop}
\begin{proof}
  そのうち執筆
\end{proof}

\begin{prop}
  $\{F_i\}_{i\in I}$を
  $\Mod(k_{X_{sa}})$のfilitrant inductive systemとし,
  $U\in\Op^c(X_{sa})$とする.
  このとき,
  \[
    \ilim[i\in I] \Gamma(U ; F_i)
    \simeq
    \Gamma(U ; \ilim[i\in I] F_i)
  \]
  が成り立つ.
\end{prop}
\begin{proof}
  $\Mod(k_{X_{sa}^c})$で示せば十分である.
  $\iilim[i] F_i$で, 
  $X_{sa}^c$上の前層
  $V\mapsto \ilim[i] \Gamma(U;F_i) = \ilim[i] ( F_i(U) )$を表す.
  $U\in\Op^c(X_{sa})$とし,
  $S$を$U$の相対コンパクトな有限被覆($\Op^c(X_{sa})$の位相での被覆$\mathscr{S}(U)$の有限被覆?)とする.
  同型
  $(\iilim[i] F_i)(S) \to \ilim[i] ( F_i(S) ) $
  が存在し,
  $F_i\in\Mod(k_{X_{sa}^c})$より,
  $(\iilim[i] F_i)(S) \simeq \ilim[i] ( F_i(U) ) =  (\iilim[i] F_i)(U)$
  が成り立つ.

  ここで, $U$の相対コンパクトな有限被覆の族$\mathscr{S}^f(U)$は, $\Cov(U)$でcofinalなので,
  $
  (\iilim[i] F_i)(U)
  \simeq \ilim[S\in\Cov^f(U)] (\iilim[i]F_i)(S)
  \simeq \ilim[S\in\Cov(U)] (\iilim[i]F_i)(S)
  = (\iilim[i] F_i)^+(U)
  $
  となる.
  よって,
  \[
    \begin{tikzcd}
      \iilim[i] F_i \ar[r,"\sim"]
      & (\iilim[i] F_i)^+ \ar[r,"\sim"]
      & (\iilim[i] F_i)^{++}
    \end{tikzcd}
  \]
  を得る.
  また, 層$\ilim[i] F_i$の定義より,
  $(\iilim[i] F_i)^{++}=\ilim[i] F_i$なので,
  $\iilim[i] F_i = \ilim[i] F_i$が言える.
  これに$\Gamma(U;\bullet)$を施せば,
  $\ilim[i]\Gamma(U;F_i) \simeq \Gamma(U;\ilim[i]F_i)$となる.
\end{proof}

\begin{prop}
  $F\in\Psh(X_{sa}^c)$が, 次の2条件を満たすとする.
  \begin{enumerate}[(i)]
    \item $F(\emptyset)=0$,
    \item 任意の$U,V\in\Op^c(X_{sa})$に対して,
      \[
        \begin{tikzcd}
          0 \ar[r]
          & F(U \cup V) \ar[r]
          & F(U) \oplus F(V) \ar[r]
          & F(U \cap V)
        \end{tikzcd}
      \]
      は完全列である.
  \end{enumerate}
  このとき,
  $F\in\Mod(k_{X_{sa}^c})\simeq\Mod(k_{X_{sa}})$である.
\end{prop}
\begin{proof}
  $U\in\Op^c(X_{sa})$と$U$の有限被覆$\{U_j\}_{j=1}^{n}$とする.
  また, $U_{ij} = U_i \cap U_j$と略記する.
  次の列が完全であることを示せば良い.
  \[
    \begin{tikzcd}
      0 \ar[r]
      & F(U) \ar[r]
      & \bigoplus_{1 \leq k \leq n} F(U_k) \ar[r]
      & \bigoplus_{1 \leq i < j \leq n} F(U_{ij})
    \end{tikzcd}.
  \]
  ただし, 2番目の射は,
  $(s_k)_{1 \leq k \leq n}$を,
  $(\restr{s_i}{U_{ij}}-\restr{s_j}{U_{ij}})_{1 \leq i < j \leq n}$
  に対応させる射である.

  $n$についての帰納法で示す.
  $n=1$は明らかで, $n=2$は仮定の(\rnum{2})そのものである.
  $1 \leq j \leq n-1$では成り立つと仮定する.
  また, $U^\prime = \bigcup_{1\leq k \leq n-1} U_k$と略記する.
  このとき, 帰納法の仮定から, 完全列による可換図式
  \[
    \begin{tikzcd}
      && 0 \ar[d] & 0 \ar[d]
      \\
      0 \ar[r]
      & F(U) \ar[r]
      & F(U^\prime) \oplus F(U_n) \ar[r] \ar[d]
      & F(U^\prime\cap U_n) \ar[d]
      \\
      &
      & \bigoplus_{1 \leq k \leq n-1} F(U_k) \oplus F(U_n) \ar[r] \ar[d]
      & \bigoplus_{1 \leq i \leq n-1} F(U_{in})
      \\
      &
      & \bigoplus_{1 \leq i < j \leq n-1} F(U_{ij})
      &
    \end{tikzcd}
  \]
  を得る.
  この図式で diagram chasing を行えば, 目的の結果を得る.
\end{proof}
\end{document}
