\documentclass[../main]{subfiles}

\begin{document}
\setcounter{section}{2}
\setcounter{subsection}{2} \addtocounter{subsection}{-1}
\subsection{Subanalytic sites and sheaves}
\begin{dfn}[subanalytic site]
  $\Op(X_{sa})$を$X$のsubanalyticな部分集合の圏とする.
  この圏には, 次のような位相が入る:

  $S\subset\Op(X_{sa})$が$U\in\Op(X_{sa})$の被覆であるとは,
  $X$の任意のコンパクト集合$K$に対して,
  ある有限部分集合$S_0\subset S$で,
  $K\cap (\cup_{V\in S_0} V) = K\cap U$
  となるものが存在する.

  このような$X_{sa}$をsubanalytic siteと言う.
  また, $U_{X_{sa}}$を,
  $\Op(X_{sa})\cap U$に$X_{sa}$の位相から誘導される位相を入れたものとする.
  一般に, $U_{sa}$と$U_{X_{sa}}$は異なる.
\end{dfn}

\begin{eg}
  $X = \mathbb{R}^2$, $U = \mathbb{R}^2\setminus\{0\}$とする.
  このとき, $V_n = \left\{ x\in\mathbb{R}^2 \relmiddle{|} |x|>\frac{1}{n} \right\}$とすると,
  $\{V_n\}_{n\in\mathbb{N}}\in\Cov(U_{sa})$であるが,
  $\{V_n\}_{n\in\mathbb{N}}\not\in\Cov(U_{X_{sa}})$である.
\end{eg}
\end{document}
